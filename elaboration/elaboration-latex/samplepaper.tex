% This is samplepaper.tex, a sample chapter demonstrating the LLNCS macro
% package for Springer Computer Science proceedings; Version 2.21 of 2022/01/12
%
\documentclass[runningheads]{llncs}
%
\usepackage[T1]{fontenc}
\usepackage{fontspec}
\setmonofont{Noto Sans Mono}[Scale=0.85]
\usepackage{amssymb}
\usepackage{amsmath}
\usepackage{multicol}
\usepackage{xcolor}
\usepackage{soul}
\newcommand{\mathcolorbox}[2]{\colorbox{#1}{$\displaystyle #2$}}
\definecolor{hlgray}{gray}{0.95}
\usepackage{listings}
\lstset{language=Haskell, basicstyle=\ttfamily}
\usepackage{hyperref}

% T1 fonts will be used to generate the final print and online PDFs, so please
% use T1 fonts in your manuscript whenever possible. Other font encondings may
% result in incorrect characters.
%
\usepackage{graphicx}
% Used for displaying a sample figure. If possible, figure files should be
% included in EPS format.
%
% If you use the hyperref package, please uncomment the following two lines to
% display URLs in blue roman font according to Springer's eBook style:
% \usepackage{color} \renewcommand\UrlFont{\color{blue}\rmfamily}
%
\begin{document}
%
\title{The Problem with "Type in Type" and a resolution thereof}
%
%\titlerunning{Abbreviated paper title} If the paper title is too long for the
% running head, you can set an abbreviated paper title here
%
\author{Tobias Hoffmann}
%
\authorrunning{T. Hoffmann}
% First names are abbreviated in the running head. If there are more than two
% authors, 'et al.' is used.
%
\institute{ Albert-Ludwigs-Universität Freiburg\\
\email{garbaz@t-online.de}}
%
\maketitle              % typeset the header of the contribution
%
\begin{abstract}
We explore the derivation of a known paradox which arises from the assumption
"type in type" in a dependently typed lambda calculus, showing therefore the
inconsistency of such a type system. We present a simple modification to the
type rules which restores consistency, and provide an implementation of typing
for this adapted dependently typed lambda calculus.
\keywords{Type theory \and Dependent types \and Girard's paradox}
\end{abstract}
%
%
%
\section{Overview}

In \textit{"A Tutorial Implementation of a Dependently Typed Lambda Calculus"}
(A. Löh et al., 2010) \cite{loh2010tutorial} a basic dependently typed lambda
calculus and an implementation in Haskell of type inference and checking for it
are presented. The type system chosen for this dependently typed lambda
calculus, which we shall call $\lambda_{\Pi}^{\tau \tau}$, and which will be
taken as the basis of the following discussion, is a direct extension of the
simply typed lambda calculus (STLC), with function types, $\tau \rightarrow
\tau'$, extended to dependent function types, $\forall x . \rho . \rho'$, and
the distinction between ordinary terms and type terms being dissolved. From this
results the necessity for a term which is the type of all types, $\ast$, which
in turn of course also requires a type itself. The perhaps most straightforward
choice to be made here is to consider the type of $\ast$ to be $\ast$ itself
("type in type"). Just like in Martin-Löf's 1971 "A Theory of Types"
\cite{martin1971theory} this is in fact the choice that A. Löh et al. made.
However, contrary to Martin-Löf in 1971, they did so in full knowledge that this
results in an inconsistent type system, as was shown by Girard in 1972
\cite{girard1972interpretation}, to keep the type system and it's implementation
simple.

The inconsistency arising from $\ast : \ast$ will be the focus of the first part
of this paper. Though instead of Girard's original proof, a much simplified
construction due to Hurkens \cite{hurkens1995simplification} will be discussed,
which in the following shall be called "Hurkens' paradox". From this paradox we
will arrive at a (relatively) compact concrete term of type $\forall A: \ast. A$,
which should be impossible in a consistent type system, given that by applying
this term to any possible type we receive a term of that type, including any
definitionally empty types. So, looked at through the Curry-Howard
correspondence, where we take types to represent propositions and terms to
represent proofs, this entails that we can provide a proof for every possible
proposition, which clearly would make such an implementation of little use as
the basis for a proof assistant.

While there are different ways of resolving the inconsistency arising from $\ast
: \ast$, in the second part of this paper, one possible solution, namely a
"hierarchy of sorts" will be introduced, and an extended version of the lambda
calculus and typing implementation presented by A. Löh et al., which we shall
call $\lambda_{\Pi}^{\omega}$, will be presented.

For a complete and annotated Agda source file implementing Hurkens' paradox, a
translation thereof into the abstract syntax of an implementation of
$\lambda_{\Pi}^{\tau \tau}$ in Haskell, and the implementation of
$\lambda_{\Pi}^{\omega}$, refer to the code repository associated with this
paper \cite{GitHubGa38:online}.

\section{Hurkens' Paradox}

In 1995, Antonius J.C. Hurkens derived, based upon work by Girard
\cite{girard1972interpretation} and Coquand \cite{coquand1986analysis}, a
relatively compact term of type $\bot$ in $\lambda U^-$
\cite{hurkens1995simplification}. While the type system of $\lambda U^-$ goes
beyond the type system of $\lambda_{\Pi}^{\tau \tau}$, his construction can be
followed one-to-one, giving us a term of type $\bot$ in $\lambda_{\Pi}^{\tau
\tau}$, proving the type system's inconsistency, which we shall do in the
following.

Though Hurkens showed two different approaches to simplifying Girard's paradox,
the one for which he provided a complete term of type $\bot$ is based upon the
concept of "powerful universes", and will be the one explored here.

While the goal in the end will be to construct the paradox in the mentioned
implementation of $\lambda_{\Pi}^{\tau \tau}$, for readability and convenience,
in the following, the syntax of the dependently typed programming language Agda
\cite{TheAgdaW80:online} will be used in the explanation of the paradox. Also,
each type theoretic definition and proof, given in Agda syntax, will be accompanied with a
directly corresponding set theoretic elaboration of the proof.

\subsection{Basic Definitions}

In absence of record types, we will, as is common, define the empty type $\bot$
and negation $\neg$ as follows:

\begin{verbatim}
⊥ : Set
⊥ = (A : Set) → A

¬ : Set → Set
¬ P = P → ⊥
\end{verbatim}

A term of type $\bot$ would have to be a function that could produce a term for
any possible type, i.e. a proof of every possible proposition. Therefore this
type has to be empty for the type system to be consistent.

A term of type $\neg P$ for some proposition $P$ is simply a function which, if
a proof of $P$ were given, would produce a term of type $\bot$. Therefore, if we
can construct a term of type $\neg P$, then it follows that we can not possibly
produce a term of type $P$. At least that is the case in a consistent type
system. Therefore the proposition $P$ must not be true.

This gives us the first hint as to how we could derive a term of type $\bot$. If
we can come up with some proposition $P$ for which we can both derive a term of
type $P$ and of type $\neg P$, then we simply have to apply $\neg P$ to $P$ and
we will have our term of type $\bot$ in hand. In fact, this will be exactly what
we shall do in the end, however, first we have to come up with such a
proposition.

For ease of readability and conceptual understanding, we shall also define a
function for the type of all propositions over some type $A$. From a
set-theoretic perspective, this is to be understood as the set of all subsets
for some set $A$, i.e. it's power set:

\begin{verbatim}
℘ : Set → Set
℘ A = A → Set
\end{verbatim}

This powerset function will be made extensive use of in the following to allow
us to build up our paradox.

\subsection{A Powerful Universe}

The first significant definition for Hurkens' Paradox is an instance of a
\textit{powerful universe}, which we shall consider essentially plucked out of
thin air, in the knowledge that it will allow us to derive our contradiction:


\begin{verbatim}
U : Set
U = (A : Set) → (℘ (℘ A) → A) → ℘ (℘ A)

τ : ℘ (℘ U) → U
τ ppU A ppA→A pA = ppU (λ u → pA (ppA→A (u A ppA→A)))

σ : U → ℘ (℘ U)
σ u = u U τ        
\end{verbatim}

This triple of $(U,\sigma,\tau)$ we consider to be \textit{powerful}, since it
satsifies, in set theoretic terms, the following property:

$$
\forall C \in \wp (\wp U): \sigma (\tau C) = \left \{ X | \left \{ y | \tau (\sigma y) \in X \right \} \in C \right \}
$$

We will not concern ourselves with translating this property into type theory or
proving that this property holds for our $(U, \sigma, \tau)$ (see Hurkens'
original derivation \cite{hurkens1995simplification} for some elaboration on the
definition of a powerful universe), since such a proof term will not be
necessary in constructing our paradox. Rather we will implicitly use this
property as it arises from the behaviour of $\tau$ and $\sigma$ as defined
above.

\subsection{Inductive Subsets and Well Founded Elements}

For subsets of $U$ we define the following proposition:

\begin{verbatim}
inductive' : ℘ (℘ U)
inductive' pU = ((u : U) → σ u pU → pU u)
\end{verbatim}


In set theoretic terms this means that for some subset $pU$ of $U$, we consider
$pU$ to be \textit{inductive} iff the following property holds:

$$
\forall u \in U: \left ( pU \in \sigma u \Rightarrow u \in pU \right )
$$

Using this property over subsets of $U$, we define a proposition over elements
of $U$:

\begin{verbatim}
well-founded : ℘ U
well-founded u = (pU : ℘ U) → inductive' pU → pU u
\end{verbatim}

In set theoretic terms this means that we consider some element $u$ of $U$ to be
\textit{well founded} iff it is in every inductive subset of $U$.

\subsection{A Paradoxical Element}

With $\tau$ from our definition of $U$ as a powerful universe, we pick out a
specific element in $U$ for which we can show that it simultaneously is well
founded and isn't well founded, which will give us the contradiction we seek:

\begin{verbatim}
Ω : U
Ω = τ inductive'
\end{verbatim}

This means that in set theoretic terms, we consider $\Omega$ to be:

$$
\tau \left ( \left \{ pU \in \wp U | pU \, is \, inductive \right \} \right )
$$

\subsection{The Paradoxical Element is well founded}

The proof that $\Omega$ is well founded is relatively straightforward:

\begin{verbatim}
well-founded-Ω : well-founded Ω
well-founded-Ω pU ind-pU = ind-pU Ω (λ u → ind-pU (τ (σ u)))
\end{verbatim}

In set theoretic terms, we need to show that for any inductive subset $pU$ of
$U$, $\Omega$ is in $pU$. Knowing that $pU$ is inductive, this means that we
need to show that $pU$ is in $\sigma \Omega$.

Since $U$ is powerful, $\sigma \Omega = \left \{ pU \in \wp U | \left \{ u \in U
| \tau (\sigma u) \in pU \right \} \, is \, inductive \right \}$. So to show
that our $pU$ is in $\sigma \Omega$, we need to show that $\left \{ u \in U |
\tau (\sigma u) \in pU \right \}$ is inductive.

Since $pU$ is inductive, we known that for any $u \in U$, if $pU$ is in $\sigma
(\tau (\sigma u))$, then $\tau (\sigma u)$ is in $pU$. But this already is
exactly what we need to show to prove that $\left \{ u \in U | \tau (\sigma u)
\in pU \right \}$ is inductive, therefore our proof is complete.

\subsection{But also not well founded}

To construct the proof that $\Omega$ is at the same time not well founded, we
define one more concept:

\begin{verbatim}
_<_ : U → U → Set
v < u = (pU : ℘ U) → σ u pU → pU v
\end{verbatim}

For some $u \in U$, we consider some $v \in U$ to be a \textit{predecessor} of
$u$ iff for every subset $pU$ of $U$, $pU$ being in $\sigma u$ implies that $v$
is in $pU$.

With this concept, we define ourselves one specific subset of $U$, which will
turn out to be inductive:

\begin{verbatim}
Δ : ℘ U
Δ u = ¬ (τ (σ u) < u)
\end{verbatim}

So in set theoretic terms, $\Delta := \left \{ u \in U | \tau (\sigma u) \nless
u \right \}$.

We prove that $\Delta$ is inductive:

\begin{verbatim}
inductive-Δ : inductive' Δ
inductive-Δ u σuΔ τσu<u =
     τσu<u Δ σuΔ (λ pU → τσu<u λ w → pU (τ (σ w)))
\end{verbatim}

To show that $\Delta$ is inductive, we need to show that for any $u \in U$, if
$\Delta$ is in $\sigma u$ then $u$ is in $\Delta$, so $\tau (\sigma u) \nless
u$.

So for any $u \in U$ with $\Delta \in \sigma u$, we assume $\tau (\sigma u) < u$
and arrive at a contradiction from this assumption as follows:

$\tau (\sigma u) < u$ means that for any $pU$ in $\sigma u$, $\tau (\sigma u)$
is in $pU$. If we take for $pU$ $\Delta$ itself, this means that $\tau (\sigma
u)$ is in $\Delta$, so $\tau (\sigma (\tau (\sigma u))) \nless \tau (\sigma u)$.

On the other hand we can show that $\tau (\sigma (\tau (\sigma u))) < \tau
(\sigma u)$ as follows:

For any subset $pU$ of $U$ we have to show that if $pU$ is in $\sigma (\tau
(\sigma u))$, then $\tau (\sigma (\tau (\sigma u)))$ is in $pU$. However, since
$U$ is powerful, this simplifies to having to show that if $\left \{w | \tau
(\sigma w) \in pU \right \}$ is in $\sigma u$, then $\tau (\sigma u)$ is in
$\left \{w | \tau (\sigma w) \in pU \right \}$. But that follows directly from
our initial assumption that $\tau (\sigma u) < u$. Therefore our proof is
complete.

With $\Delta$ and the knowledge that it is inductive in hand, we can at last
prove that $\Omega$ is not well founded, which will complete the contradiction
we seek:

\begin{verbatim}
¬well-founded-Ω : ¬ (well-founded Ω)
¬well-founded-Ω wfΩ =
     wfΩ Δ inductive-Δ (λ pU → wfΩ (λ w → pU (τ (σ w))))
\end{verbatim}

To show that $\Omega$ is not well founded, we assume that it is, and will from
this derive a contradiction:

Since $\Delta$ is inductive and $\Omega$ well founded, this means that $\Omega$
is in $\Delta$, and therefore $\tau (\sigma \Omega) \nless \Omega$. On the other
hand, we can show that $\tau (\sigma \Omega) < \Omega$ as follows:

For any subset $pU$ of $U$ we have to show that if $pU$ is in $\sigma \Omega$,
then $\tau (\sigma \Omega)$ is in $pU$. However, since $U$ is powerful and
$\Omega$ was defined as $\tau \left \{ pU | pU \, is \, inductive \right \}$,
this simplifies to having to show that if $\left \{w | \tau (\sigma w) \in pU
\right \}$ is inductive, then $\Omega$ is in $\left \{w | \tau (\sigma w) \in pU
\right \}$. But that follows directly from our initial assumption that $\Omega$
is well founded. Therefore our proof is complete.

\subsection{A term of the empty type}

With both a proof that $\Omega$ is well founded and a proof that $\Omega$ is not
well founded in hand, we can at last construct the term of type $\bot$:

\begin{verbatim}
false : ⊥
false = ¬well-founded-Ω well-founded-Ω
\end{verbatim}

This concludes the proof that $\lambda_\Pi^{\tau \tau}$ is inconsistent.

\section{A Hierarchy of Sorts}

As evident from the construction of a contradiction in $\lambda_\Pi^{\tau \tau}$
presented above, it is necessary for the expressiveness of our dependently typed
lambda calculus to be weakened in some way for it to be consistent. However, we
simultaneously do not want to give up the ability to express propositions of
practical interest and their proofs in our lambda calculus.

Luckily, a rather simple modification to the type system of $\lambda_\Pi^{\tau
\tau}$ is sufficient to make it consistent again \cite{lof1973intuitionistic}
\cite{coquand1986analysis}, replacing the problematic type rule $\ast : \ast$ by
a \textit{hierarchy of sorts}:

$$
\begin{aligned}
\ast   & : \ast_1\\
\ast_1 & : \ast_2\\
\ast_2 & : \ast_3\\
\ast_3 & : \ast_4\\
& ...
\end{aligned}
$$

So just like in $\lambda_\Pi^{\tau \tau}$ every object term, like $true$, has
some type, like $Bool$, and every type term has the kind $\ast$. However, the
term $\ast$ itself does not have the kind $\ast$, but the \textit{sort}
$\ast_1$, the term $\ast_1$ has the sort $\ast_2$, the term $\ast_2$ has the
sort $\ast_3$ and so on.

In the following, we will write for consistency of notation $\ast_0$ for the
sort of types, istead of $\ast$.

This lambda calculus we shall call $\lambda_\Pi^\omega$. The grammar and type
rules for it are as follows:

$$
\begin{aligned}
e , \rho ::= & \quad e : \rho\\
           | & \quad x\\
           | & \quad e \; e'\\
           | & \quad \lambda x . e\\
           | & \quad \ast_\ell\\
           | & \quad \forall x . \rho . \rho'
\end{aligned}
$$

\begin{multicols}{3}

$$
\frac{\Gamma \vdash e :_\uparrow \tau}
     {\Gamma \vdash e :_\downarrow \tau}
$$

$$
\frac{\Gamma(x) =  \tau}
        {\Gamma \vdash x :_\uparrow \tau}
$$

$$
\frac{\Gamma , x : \tau \vdash e :_\downarrow \tau'}
     {\Gamma \vdash \lambda x . e :_\downarrow \forall x . \tau . \tau'}
$$



\end{multicols}
\begin{multicols}{2}

$$
\frac{\Gamma \vdash e :_\uparrow \forall x . \tau . \tau' \quad \Gamma \vdash e' :_\downarrow \tau}
     {\Gamma \vdash e \; e' :_\uparrow \tau' \! \left [ \, x \mapsto e' \, \right ]}
$$

$$
\frac{\Gamma \vdash \mathcolorbox{hlgray}{\rho :_\uparrow \ast_\ell} \quad \rho \Downarrow \tau \quad \Gamma \vdash e :_\uparrow \tau}
     {\Gamma \vdash (\,e : \rho\,) :_\uparrow \tau}
$$

\end{multicols}
\begin{multicols}{2}

$$
\frac{\Gamma \vdash \mathcolorbox{hlgray}{\rho :_\uparrow \ast_\ell} \quad \rho \Downarrow \tau \quad \Gamma , x : \tau \vdash \mathcolorbox{hlgray}{\rho' :_\uparrow \ast_{\ell'}}}
     {\Gamma \vdash \mathcolorbox{hlgray}{\forall x . \rho . \rho' :_\uparrow \ast_{max(\ell, \ell')}}}
$$

$$
\frac{}
     {\Gamma \vdash \mathcolorbox{hlgray}{\ast_\ell :_\uparrow \ast_{\ell + 1}}}
$$

\end{multicols}

Only the last three rules differ from $\lambda_\Pi^{\tau \tau}$, and only the
last two do so substantially. Also of note is that some type judgements have
turned from type checking to type inference due to the need to know the
\textit{level} $\ell$ for a sort $\ast_\ell$, which will in turn necessitate
corresponding adaptations in the implementation.

The last type rule, $\ast_\ell : \ast_{\ell + 1}$, is a direct implementation of
the hierarchy of sorts as explained above.

But of course it is also necessary to reconsider type judgements over terms
$\forall x . \rho . \rho'$ in the second to last type rule. 

One option in defining the type rule for $\forall x . \rho . \rho'$ would be to
simply directly keep the rule from $\lambda_\Pi^{\tau \tau}$ in our new type
system:

$$
\frac{\Gamma \vdash \rho :_\downarrow \ast_0 \quad \rho \Downarrow \tau \quad \Gamma , x : \tau \vdash \rho' :_\downarrow \ast_0}
     {\Gamma \vdash \forall x . \rho . \rho' :_\uparrow \ast_0}
$$

While this would be consistent (since it is simply a strictly less powerful type
system than that of $\lambda_\Pi^\omega$), and would have the advantage of that
we would only have to check that the kind of $\rho$ and $\rho'$ is $\ast_0$, it
would greatly restrict the expressiveness of our language, since it would mean
that a function could not even take a type as an argument or return a type as it's
result, it could only go from objects to objects.

Instead, we will allow for functions to go from any sort to any sort, from
objects to types, from types to objects, from kinds to objects, from objects to
kinds, etc. :

$$
\frac{\Gamma \vdash \rho :_\uparrow \ast_\ell \quad \rho \Downarrow \tau \quad \Gamma , x : \tau \vdash \rho' :_\uparrow \ast_{\ell'}}
     {\Gamma \vdash \forall x . \rho . \rho' :_\uparrow \ast_{max(\ell, \ell')}}
$$

So the sort of some term $\forall x . \rho . \rho'$ is simply the higher of the
sorts of $\rho$ and $\rho'$.

Taking $\rho \rightarrow \rho'$ as a shorthand for $\forall x:\rho . \rho'$ with
$x$ not appearing in $\rho'$, this means for example that a term $\ast_7
\rightarrow \ast_3$ is of the sort $\ast_8$, since $\ast_7 : \ast_8$ and $\ast_3
: \ast_4$.

As a more practical example, the "power set" function $\wp$ we defined above for
our construction of Hurkens' paradox, $\wp A := A \rightarrow \ast_0$, would
have the sort $\ast_0 \rightarrow \ast_1$.

This already hints towards how the construction of Hurkens' paradox
presented above might no longer work in $\lambda_\Pi^\omega$, given that it
heavily relies on the fact that in $\lambda_{\Pi}^{\tau \tau}$, $\wp : \ast
\rightarrow \ast$.

The definition of the function $\wp$ above might also raise the question what to
do if we do not want to look at the type of all propositions over some type, but
rather over some kind, or over some higher sort. Do we define $\wp_1 A := A
\rightarrow \ast_0$ of sort $\ast_1 \rightarrow \ast_2$, $\wp_2 A := A
\rightarrow \ast_0$ of sort $\ast_2 \rightarrow \ast_3$, and so on?

In the type system presented here, we do not have any other choice but to do so.
However, there are possible extensions to alleviate this redundancy of
definitions, namely \textit{universe polymorphism} \cite{sozeau2014universe}.
However, not only does this require levels to become terms inside the lambda
calculus itself, but also brings with it the need to introduce a second
hierarchy of sorts for level polymorphic function types. We will not implement
this here.

\section{Implementation}

With the introduction of a hierarchy of sorts having necessitated that some
judgements in our type rules have turned from type checking to type inference,
the implementation of $\lambda_\Pi^{\tau \tau}$ has to be changed in quite a few
places. Though most of these changes are not terribly significant, so we will focus
here on the changes to the implementation due to the new type rules for $\,e :
\rho\,$, $\forall x . \rho . \rho'$, and $\ast_\ell$.

\subsection{Abstract Syntax}

\begin{lstlisting}
data TermInfer
  = Ann TermCheck TermInfer
  | Star Int
  | Pi TermInfer TermInfer
  | Bound Int
  | Free Name
  | TermInfer :@: TermCheck
  deriving (Show, Eq)
\end{lstlisting}

\lstinline{Star} now no longer simply is a constant constructor, but has an
argument, it's level. And both \lstinline{Ann} and \lstinline{Pi} now have in
places a \lstinline{TermCheck} replaced with \lstinline{TermInfer} due to the
necessary changes to the type rules.

\subsection{typeInfer for Ann}

\begin{lstlisting}
typeInfer i g (Ann e r) =
  do
    s <- typeInfer i g r
    case s of
      (VStar l) -> do
        let t = evalInfer [] r
        typeCheck i g e t
        return t
      _ -> failure ":("
\end{lstlisting}

We infer the type of \lstinline{r}, which has to be some \lstinline{VStar l} or
we fail. Otherwise we proceed as in the implementation of $\lambda_\Pi^{\tau
\tau}$, evaluating \lstinline{r} to \lstinline{t} and checking \lstinline{e}
against \lstinline{t}.

\subsection{typeInfer for Pi}

\begin{lstlisting}
typeInfer i g (Pi r r') =
  do
    s <- typeInfer i g r
    case s of
      (VStar l) -> do
        let t = evalInfer [] r
        s' <-
          typeInfer
            (i + 1)
            ((Local i, t) : g)
            (substInfer 0 (Free (Local i)) r')
        case s' of
          (VStar l') -> return (VStar (max l l'))
          _ -> failure ":("
      _ -> failure ":("
\end{lstlisting}

We infer the type for \lstinline{r}, which has to be some \lstinline{VStar l} or
we fail. Otherwise, we evaluate \lstinline{r} to \lstinline{t} and infer the
type of \lstinline{r'} in the extended context, which again has to be some
\lstinline{VStar l'}. With both the sort level \lstinline{l} of \lstinline{r}
and \lstinline{l'} of \lstinline{r'} determined, we can return the sort of our
\lstinline{Pi} term, \lstinline{VStar (max l l')}.

And that is all.

%
% ---- Bibliography ----
%
% BibTeX users should specify bibliography style 'splncs04'. References will
% then be sorted and formatted in the correct style.
%
% \bibliographystyle{splncs04} \bibliography{mybibliography}
%
\bibliographystyle{splncs04}
\bibliography{samplepaper}



% \begin{thebibliography}{8} \bibitem{ref_article1} Author, F.: Article title.
% Journal \textbf{2}(5), 99--110 (2016)

% \bibitem{ref_lncs1} Author, F., Author, S.: Title of a proceedings paper. In:
% Editor, F., Editor, S. (eds.) CONFERENCE 2016, LNCS, vol. 9999, pp. 1--13.
% Springer, Heidelberg (2016). \doi{10.10007/1234567890}

% \bibitem{ref_book1} Author, F., Author, S., Author, T.: Book title. 2nd edn.
% Publisher, Location (1999)

% \bibitem{ref_proc1} Author, A.-B.: Contribution title. In: 9th International
% Proceedings on Proceedings, pp. 1--2. Publisher, Location (2010)

% \bibitem{ref_url1} LNCS Homepage, \url{http://www.springer.com/lncs}. Last
% accessed 4 Oct 2017 \end{thebibliography}
\end{document}
